\documentclass[12pt,a4paper]{article}

\usepackage[utf8]{inputenc}
\usepackage[ngerman]{babel}
\usepackage[T1]{fontenc}
\usepackage{amsmath}
\usepackage{amsfonts}
\usepackage{amssymb}
\usepackage{graphicx}
\usepackage[left=2cm,right=2cm,top=2cm,bottom=2cm]{geometry}
\usepackage{multicol}
\usepackage{booktabs}
\usepackage[hidelinks]{hyperref}
\usepackage{tikz}
\usepackage{pgfplots}
\usepackage{blindtext}
\usepackage{array}
\usepackage{multirow}
\usepackage{bigdelim}
\usepackage{colortbl}
\usepackage{fancyhdr} 
\usepackage{tabularx}
\usepackage{pgfplots}
\usepackage{xcolor}
\usepackage{color}
\usetikzlibrary{decorations.text}
\usetikzlibrary{tikzmark}
\pagestyle{fancy} 
	\fancyhf{} 
	\fancyhead[L]{\includegraphics[scale=0.05]{Bilder/dhbw.png}} 
	\fancyhead[C]{\slshape Statistik} 
	\fancyhead[R]{\slshape LaTeX Version}

\usepackage{helvet}
\renewcommand{\familydefault}{\sfdefault}

\newcolumntype{Z}{>{\centering\let\newline\\\arraybackslash\hspace{0pt}}X}
\author{\slshape Robin Rausch, Florian Maslowski, Ozan Akzebe}
\title{Statistik}
\date{\slshape \today}
\begin{document}
\maketitle
\tableofcontents
\newpage
\section{Statistik Grundlagen und Definitionen}
\subsection{Arithmetischer Mittelwert}
Der Arithmetische Mittelwert $\bar{x} $ ist der \glqq normale\grqq{} Durchschnitt. Also addiert man alle Werte auf und teilt durch die Anzahl der Werte. Beispiel:\\
\begin{equation}
	\frac{1 + 2 + 2 + 1 + 4}{5} = 2
\end{equation}

\subsection{Grundbegriffe}
\subsubsection{Grundgesamtheit}
Beschreibt die Menge gleicherartiger Objekte (z.B. Mietwohnungen).

\subsubsection{Stichprobe}
Beschreibt die Menge der untersuchten Objekte (z.B. Mietwohnungen in Stuttgart).

\subsubsection{Stichprobenumfang}
Beschreibt die Anzahl der untersuchten Objekte (z.B. 120 Mietwohnungen in Stuttgart).

\subsubsection{Merkmal}
Beschreibt ein Kriterium oder eine interessierende Größe (z.B. Anzahl der Zimmer).

\subsubsection{Ausprägung}
Beschreibt die Werte, die ein Merkmal annehmen kann (z.B. 1 Zimmer, 2 Zimmer, 4 Zimmer, \dots).

\subsection{Qualitativ und Quantitativ}
\subsubsection{Qualitativ}
Merkmale heißen qualitativ, wenn sie nur ihrer \textcolor{blue}{Art} nach unterschieden werden können.

\subsubsection{Quantitativ}
Merkmale heißen quantitativ, wenn sie nur ihrer \textcolor{blue}{Größe} unterscheidbar sind.

\subsection{Diskret und Stetig}
\subsubsection{Diskret}
Merkmale heißen diskret, wenn sie \textcolor{blue}{abzählbar} sind.

\subsubsection{Stetig}
Merkmale heißen stetig, wenn sie \textcolor{blue}{jeden Wert innerhalb eines Intervalls} annehmen können.

\subsection{Skalierung von Merkmalen}
\subsubsection{nominal}
Die Skalierung von Merkmalen heißt nominal, wenn sie in verschiedener Reihenfolge aufgelistet werden können (z.B. Augenfarbe).

\subsubsection{ordinal}
Die Skalierung von Merkmalen heißt ordinal, wenn sie in auf- oder absteigender Reihenfolge aufgezählt werden kann (z.B. Wohnungsgröße).

\subsubsection{kardinal}
Die Skalierung von Merkmalen heißt kardinal, wenn bei auf- oder absteigender Reihenfolge Abstände definiert werden können (z.B. Geschosshöhe).

\subsection{Häufigkeit}
Die Häufigkeit $h_i$ ist die Anzahl der jeweils $i$-ten Ausprägung eines untersuchten Merkmals/Ereignisses. Beispiel:\\
Wenn es bei einem Experiment 11 Proband*innen gibt und 3 davon die Haarfarbe braun haben, ist die Häufigkeit der Haarfarbe braun 3. Die Häufigkeit aller Haarfarben kombiniert wäre dann 11.

\subsection{relative Häufigkeit}
Die relative Häufigkeit $f_i$ ist die normale Häufigkeit im Verhältnis zur Gesamtanzahl. Also im o.g. Beispiel:\\
Die relative Häufigkeit der Haarfarbe braun $f_{braun} = \frac{3}{11}$.

\subsection{Häufigkeitsverteilung}
Eine Häufigkeit $h_i$ ist die Anzahl von Ausprägungen (oder Ereignissen), also $\geq 1$. Eine Häufigkeitsverteilung ist die Darstellung (=Zusammenschau) der Häufigkeiten aller möglichen Ereignissen. Also ein Diagramm oder eine Auflistung aller Häufigkeiten zur Übersicht ihrer Verteilungen zu einander. 

\subsection{Verteiung der relativen Häufigkeiten}
Die Verteilung der relativen Häufigkeit ist die Zusammenschau der relativen Häufigkeiten $f_i = \frac{h_i}{n}$ ($n$ = Stichprobenumfang und $h_i$ = Häufigkeiten) von sämtlichen möglichen Ereignissen (analog zur Häufigkeitsverteilung).

\subsection{Empirische Verteilungsfunktion}
Die Empirische Verteilungsfunktion ist die Funktion der kumulierten relativen Häufigkeiten. Beispiel:\\
\begin{equation}
	F(x) = \sum_{i}^{} f_i  
\end{equation}
\includegraphics[width=\textwidth]{Bilder/empirische_verteilungsfunktion.png}

\subsection{Modalwert}
Der Modalwert ist der Messwert eines Datensatzes, welcher am häufigsten auftritt.

\subsection{Median}
Der Medianm $\tilde{x}$ ist der Wert, der genau in der Mitte eines \textbf{geordneten} Datensatzes liegt. Beispiel:\\
Datensatz = \{ 1, 2, 2, 4, 70\}\\
Median = 2

\subsection{Kennwerte}
\subsubsection{Quantil und Quartil}
Der Median ist ebenfalls ein Quantil:\\
$x_{0,5} = \tilde{x} $\\
Das Quantil kann aber auch ein Quartil oder anderes sein:\\
unteres Quartil: $x_{0,25}$\\
oberes Quartil: $x_{0,75}$\\
\begin{equation}
	x_p = x_{0,75} = 0,5(x_8 + x_9) = \frac{1}{2} (3 + 3) = 3 = x_{0,75} = q_{obere}
\end{equation}

\subsubsection{Decil}
Als Decil werden alle Quantile mit $p = k * 0,1; k = 1; 2; ...; 10 $\\
Also $\{ x_{0,1}; x_{0,2}; x_{0,3};$ ...$; x_{0,8}; x_{0,9}; x_{1,0}; \}$

\subsubsection{Percentil}
Als Percentil werden alle Quantile mit $p = k * 0,01; k = 1; 2; ...; 100$\\
Also $\{ x_{0,01}; x_{0,02}; x_{0,03};$ ...$; x_{0,98}; x_{0,99}; x_{1,0}; \}$

\subsection{Boxplot}
\includegraphics[width=\textwidth]{Bilder/boxplot.png}
Whisker sind minimale und maximale Werte.

\subsection{Empirische Varianz}
Diese gibt an, wie stark die einzelnen Werte um den Mittelwert abweichen.

\subsection{Empirische Standardabweichung}
Die empirische Standardabweichung SD(x) lässt sich aus der empirischen Varianz berechnen:\\
\begin{equation}
	SD(x) = s = \sqrt{VAR} = \sqrt{s^2} = \sqrt{\frac{1}{n - 1} \sum_{i = 1}^{n} (x_i - \tilde{x})^2}  
\end{equation}
\textcolor{lightgray}{Immer positiv!}

\end{document}
